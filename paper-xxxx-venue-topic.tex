\documentclass[final,5pt,times,twocolumn]{elsarticle}

    \journal{Github}

    % Packages
    \usepackage{paper-xxxx-venue-topic}

    % Document
    \begin{document}

    \begin{frontmatter}
        \title{Latex Repository Template for Research Papers}

        \author[disi]{Matteo Magnini}
        \ead{matteo.magnini@unibo.it}

        \affiliation[disi]{
            organization={Department of Computer Science and Engineering (DISI), Alma Mater Studiorum -- Università di Bologna},
            addressline={via dell'Università 50},
            city={Cesena, FC},
            postcode={47522},
            country={Italy}
        }

        \begin{abstract}
            This is a template for research papers in \LaTeX.
            %
            It is based on the \texttt{elsarticle} class, so if you are submitting to a journal or conference that uses a different class, you will need to modify it accordingly.
            %
            This template include automatic release of the compiled PDF on GitHub Actions.
            %
            It also performs automatic checks on the bibliography.
            %
            For this reason, this template cites some of my papers like~\cite{DBLP:conf/atal/MagniniCO22,DBLP:journals/logcom/MagniniCO23}.
            %
            With this repository, I hope to make the life of researchers a bit easier.
            %
        \end{abstract}

        \begin{keyword}
            continuous integration \sep \LaTeX \sep GitHub Actions \sep research paper \sep healthy research
        \end{keyword}

    \end{frontmatter}

    \section{Introduction}\label{sec:introduction}
        %
        This template is structured as follows:
        %
        \begin{itemize}
            \item[] \texttt{paper-xxxx-venue-topic.tex} is the main file, where you can write the content of your paper.
            %
            Change the name of this file to match the year (xxxx), venue, and topic of your paper.
            %
            This is done automatically by running the script \texttt{rename.rb}.
            %
            \item[] \texttt{paper-xxxx-venue-topic.sty} is the style file.
            %
            Also change the name of this file to match the year (xxxx), venue, and topic of your paper.
            %
            \item[] \texttt{biblio.bib} is the bibliography file.
            %
            Do not change the name of this file.
            %
            \item[] \texttt{paper-xxxx-venue-topic.pdf} is the compiled PDF.
            %
            You can download it from the Actions tab in your repository.
            %
            \item[] \texttt{figures} is the directory where you can store your figures.
            %
            \item[] \texttt{tables} is the directory where you can store your tables.
            %
        \end{itemize}
    %
    Some other random citations~\cite{DBLP:conf/percom/MontagnaAFPKUM24,DBLP:conf/dlog/MagniniOS24}

    \section{Quirks}\label{sec:quirks}
        %
        There are three main features of this template: i) automatic release of the compiled PDF on GitHub Actions, ii) automatic checks and fixes on the bibliography, and iii) checks on acknowledgements.
        %
        \subsection{Automatic Release}\label{subsec:automatic-release}
            %
            The github workflow is defined in the file \texttt{.github/workflows/build-and-deploy-latex.yml}.
            %
            If everything is fine, it generates two compiled files: paper-xxxx-venue-topic.pdf and paper-xxxx-venue-topic-camera-ready.pdf.
            %
            The second file is the same as the first one, but it has been processed by the script \texttt{check\_acknowledgements.rb}.
            %
            This script:
            %
            \begin{itemize}
                %
                \item checks if the string ``\% NO ACKNOWLEDGEMENTS \%'' is present in the acknowledgements section.
                %
                If it is, the script just copies the file to the camera-ready version;
                %
                \item if the file has no acknowledgements section, it prints an error message and stops;
                %
                \item if the file has an acknowledgements section, it prints a warning message and copies the file to the camera-ready version.
                %
                \item if the file has a commented acknowledgements section, it copies the file to the camera-ready version without the comments.
                %
            \end{itemize}

        \subsection{Automatic Bibliographic Checks and Fixes}\label{subsec:automatic-checks-and-fixes-on-the-bibliography}
            %
            The bibliography is checked before the release of the PDF.
            %
            The checks are performed by the Ruby script \texttt{bibtex\_prettifier.rb}.
            %
            What happens is the following:
            %
            \begin{itemize}
                %
                \item the script checks if there are duplicate keys in the bibliography, and it removes them;
                %
                \item it also checks if a bibtex entry has both \texttt{doi} and \texttt{url} fields, and it removes the \texttt{url} field (this is done because it is redundant and many bibtex styles render two links, which is not good).
                %
                \item if a bibtex entry has no \texttt{doi} or \texttt{url} fields, the script prints a warning message if the comment \texttt{\% ALLOW NO DOI OR URL\%} is present in the .tex file, otherwise the CI will fail with error.
            \end{itemize}

    \section{Conclusion}\label{sec:conclusion}
    %
    Before starting to write your paper, you should run the rename.rb script.
    %
    This script renames the \texttt{paper-xxxx-venue-topic.tex} and \texttt{paper-xxxx-venue-topic.sty} files to the new file name that you must provide as input argument to the script.
    %
    It also automatically updates all the references in the other files.
    %
    Beware that the script will also remove the commented acknowledgements section in the \texttt{paper-xxxx-venue-topic.tex} file.
    %
    In this way the \textbf{CI will fail} if you do not provide an -- possibly commented -- acknowledgements section or you do not provide the string ``\% NO ACKNOWLEDGEMENTS \%'' in the acknowledgements section.
    %
    Finally, the script will delete itself.

    This is a must-read paper~\cite{DBLP:journals/csur/CiattoSAMO24} if you are into neuro-symbolic AI.

%    \section{Acknowledgements}\label{sec:acknowledgements}
%        %
%        This work has been partially supported by XXX.

    \nocite{*}
    \bibliographystyle{elsarticle-num}
    \bibliography{biblio}

    \end{document}